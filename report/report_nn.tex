\documentclass{article}

% NIPS style
\usepackage{nips12submit_e, times}

% figures
\usepackage{graphicx}
\usepackage{subfigure} 

% links
\usepackage[hidelinks]{hyperref}

% better tables
\usepackage{booktabs}
\usepackage{multirow}
\newcommand{\ra}[1]{\renewcommand{\arraystretch}{#1}}


\title{Semi-Supervised Recursive Autoencoders}


\author{
Adrian Guthals \\
\texttt{aguthals@cs.ucsd.edu} \\
\And
David Larson \\
\texttt{dplarson@ucsd.edu} \\
}

\newcommand{\fix}{\marginpar{FIX}}
\newcommand{\new}{\marginpar{NEW}}

% don't show line numbers
\nipsfinalcopy 


\begin{document}

\maketitle


\begin{abstract}
Recreate results of Socher et al. "Semi-Supervised Recursive Autoendcoders". Learning meanings of sentences using Feed Forward Neural Networks with Backpropogation.
\end{abstract}



%-----------------------------------------------------------------------------
% INTRO
%-----------------------------------------------------------------------------
\section{Introduction}

Socher et al. presented a semi-supervised method for learning meanings of sentences using recursive autoencoders \cite{Socher}.

The lecture notes state blah \cite{CSE250B}.


\begin{table}[ht]
    \centering

    \caption{The description of the table shown. What does it look like if it's two lines tall? I wonder I wonder I wonder I wonderrrrrrrrrrrr} 
    \label{tab:datasets}

    \ra{1.2}
    \begin{tabular}{@{} l l l l l @{}}
        \\
        \toprule
        \bf{Dataset} & \bf{Documents} & \bf{Vocabulary} & $\alpha$ & $\beta$ \\
        \midrule
        Classic400 & 400 & 6205 & 0.01 & 0.1 \\
        KOS        & 400 & 6906 & 0.01 & 0.1\\
        \bottomrule
    \end{tabular}
\end{table}



%-----------------------------------------------------------------------------
% BIBLIOGRAPHY
%-----------------------------------------------------------------------------

\small{
\bibliographystyle{IEEEtran}
\bibliography{sources}
}


\end{document}
